\documentclass{letter}
\makeatletter
\let\@texttop\relax
\makeatother

\topmargin      0.0in
\headheight     0.0in
\headsep        0.0in
\oddsidemargin  0.0in
\evensidemargin 0.0in
\textheight     9.0in
\textwidth      6.5in

\begin{document}
\address{Nathanael Thompson\\1102 Lincolnshire Drive\\Champaign, IL 61821\\nat@alumni.brown.edu\\(469)
230-7125 (Mobile)}
\signature{Nathanael Thompson}
\begin{letter}{Cool Software Company\\Beautiful Address\\Somewhere, CO}

\opening{Dear Sir or Madam,}
The passionate software engineer is a modern day craftsman.  
He blends logic and
art to produce functional and beautiful wares.  
To truly excel, a software craftsman must have technical expertise, passion for
the craft and a desire to continually refine and improve.
The impact is widespread.
Customers benefit because software is easier to use and more stable.
Co-developers benefit from better design, simpler integration and reduced
maintenance.  The employer benefits from improved efficiency, shorter
development cycles and higher morale.
In all these things I have strenuously applied myself through many
years of education and professional experience in order to become an excellent
software craftsman.

The primary task of any software engineer is to write software that behaves as
the customer desires.  I love writing software that works.  The foundation for
working software is extensive testing:
Such a goal demands testing of all varieties:
static code analysis,
unit testing, integration testing, end-to-end testing. 
Over the last few years I have learned, taught and championed modern development
practices around software testing, including test-driven development, test
automation/continuous integration and integration of various static analysis
tools.  
Within our media team, the core feature of our software,
I have experienced great response from my co-developers and greatly
increased efficiency in my own work.
Such a vision also
demands forethought, planning and clear communication between customer and
developer and between developers themselves.
Having led several features through their lifecycle and managing projects I have
practiced requirement gathering, specification and project tracking and the use
of a common language amongst developers centering around design patterns and
agreed-upon standards.

The proud craftsman, however, does not settle for functionality only.  Good
code construction is also an art.  Beautiful code is clean and intuitive.  It is
unified and minimal.  Like a good novel each piece flows together to create the
whole, each part serving its purpose in the story and no part superfulous.
When designing software I experience a thrill
when I see the different parts of a system and how they fit,  
to find the patterns that bring each piece together.  
I love refactoring and re-architecting software to eliminate waste and improve
performance and make it easy to modify.  Creating code that is structured,
encapsulated and well designed is ultimately satisfying.
Well designed software is isolated, simple and logical.
I write code to be read and hopefully enjoyed, focusing on naming
and style and relying on known conventions.
My efforts have been well-rewarded through multiple promotions and the
satisfaction of our company being purchased by Citrix Systems in 2012.

I would benefit your company specifically.

I think I personally would benefit working at your company for some reasons...

Thank you for your consideration.

\closing{Sincerely,}
\end{letter}
\end{document}
