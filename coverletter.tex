\documentclass{letter}
\makeatletter
\let\@texttop\relax
\makeatother

\topmargin      0.0in
\headheight     0.0in
\headsep        0.0in
\oddsidemargin  0.0in
\evensidemargin 0.0in
\textheight     9.0in
\textwidth      6.5in

\date{}
\begin{document}
\address{Nathanael Thompson\\1102 Lincolnshire Drive\\Champaign, IL 61821\\nat@alumni.brown.edu\\(469)
230-7125 (mobile)}
\signature{Nathanael Thompson}
\begin{letter}{Kapost\\Boulder, CO}

\opening{Dear Sir or Madam,}
The passionate software engineer is a modern day craftsman.  He
blends logic and art to produce functional and beautiful wares.
The impact is widespread.  Customers benefit because software is
easier to use and more stable.  Co-developers benefit from better
design, simpler integration and reduced maintenance.  The employer
benefits from improved efficiency, shorter development cycles and
higher morale.  To truly excel, a software craftsman must have
technical expertise, passion for the craft and a desire to continually
refine and improve.

In all these things I have strenuously applied myself to become an
excellent software craftsman.  My technical expertise was founded
during my undergraduate and doctoral studies where I focused on
mobile computing and distributed systems.  In addition, six years
in industry have built the skills necessary for daily code construction.
When designing software I experience a thrill when I can identify
the different parts of a system and see how they fit, searching for
the patterns that bring each piece together.  I love refactoring
and (re-)architecting software to eliminate waste, improve performance
an ease modification.  I also love creating code that is simple,
encapsulated and logical.  I write code to be read and hopefully
enjoyed, focusing on naming and style and relying on known conventions.
My efforts have been well-rewarded through multiple promotions and
the satisfaction of our company being purchased by Citrix Systems
in 2012.

It was my love for refactoring that motivated the biggest improvement
in my work style.  A desire for constant refactoring drove me to
adopt automated testing and static code analysis in my own projects,
eventually teaching myself test-driven development practices.  With
my own efficiency gains in hand, over the last few years I have
taught and championed modern development techniques within our team.
I have experienced great response from my co-developers.  From there
I have been given responsibility to manage small teams of developers
and lead several features through their life-cycle.  In these new
roles I have learned the importance of clear communication in
requirement and design description as well as the importance of a
common language between developers centered around design patterns
and agreed-upon standards.

As I have discovered my passion for software engineering it has
provoked in me a dedication to the craft.  From exploring the Kapost
website I believe my dedication to becoming a better craftsman would
allow me to integrate well into the Kapost team.  I personally would
benefit from joining such a team which shares the same values.  I
also would enjoy the opportunity to work in a new market and explore
new technologies.

Thank you for your consideration.

\closing{Sincerely,}
\end{letter}
\end{document}
