\documentstyle[centered,overlapped]{res}

\newenvironment{inmargins}[1]{\begin{list}{}{
   \leftmargin=#1 \rightmargin=#1 \parsep=0pt
   \partopsep=0pt}\item[]}{\end{list}}

\newenvironment{packed_enum}{
\begin{enumerate}
  \setlength{\itemsep}{.80pt}
  \setlength{\parskip}{0pt}
  \setlength{\parsep}{0pt}
}{\end{enumerate}}

\newenvironment{packed_item}{
\begin{itemize}
  \setlength{\itemsep}{.80pt}
  \setlength{\parskip}{0pt}
  \setlength{\parsep}{0pt}
}{\end{itemize}}

\addtolength{\textheight}{15mm}
\addtolength{\topmargin}{-5mm}
%\addtolength{\oddsidemargin}{-.25in}
%\addtolength{\textwidth}{.25in}
\addtolength{\sectionskip}{-4mm}
%\addtolength{\resumewidth}{.25in}
\addtolength{\sectionwidth}{-.3in}

%\topmargin      -0.25in
%\headheight     0.0in
%\headsep        0.0in
\oddsidemargin  -0.25in
\evensidemargin -0.25in
\resumewidth	7.0in
%\textheight     9.5in
%\textwidth      6.5in

\name{\Large{\textbf{Nathanael Thompson}}}
\address{(469) 230-7125\\nat$@$alumni.brown.edu}
\address{1102 Lincolnshire Drive\\Champaign, IL 61821\\USA}

\begin{document}
\begin{resume}
%\Resume

%
% Overview
%
\iffalse
\section{Overview}
My research focuses on the areas of mobile systems and wireless
networking as well as distributed computing and the Internet.
My objective is to apply my expertise in these areas to develop and deploy new
software systems that improve end-user communication and expand it in new
directions.
\fi

%My research is targeted at the unique systems problems found in the deployment
%of end-user created wireless networks.  Such networks are becoming increasingly
%common due the high availability of wireless networking components and
%include, for example, open
%Wi-Fi access points, social ad-hoc networks and disruption-tolerant
%opportunistic networks.
%My work focuses on improving the availability,
%accessibility and performance of these networks by creating new local and
%distributed protocols
%and algorithms and providing real world implementations that inter-operate with
%existing systems.  
%I also have two years industry experience debugging Sun's Java implementation
%and interacting with external developers.

%
% Education
%
\section{Education}

\begin{tabular*}{6.10in}[t]{l@{\extracolsep{\fill}}r}
\textbf{PhD}, Computer Science&University of Illinois at
Urbana-Champaign\\
2010&Urbana, IL\\
\end{tabular*}
\begin{packed_item}
\item
Thesis title: ``Opportunistic Resource
Management to Improve Network Service Performance in User-Created
Networks''\\
Thesis advisor: Prof. Robin Kravets
\item
GPA: 3.73/4.0
\end{packed_item}


\begin{tabular*}{6.10in}{l@{\extracolsep{\fill}}r}
\textbf{Bachelor of Science with Honors}, Computer Science&Brown University\\
2001&Providence, RI\\
\end{tabular*}
\begin{packed_item}
\item
Concentration in Systems and Networking.  GPA: 3.6/4.0
\item
Senior Honors Thesis: ``Group Membership in Ad Hoc Networks''
\end{packed_item}


%
% research experience
%
\section{Professional Experience}
% ByteMobile
\begin{tabular*}{6.10in}{l@{\extracolsep{\fill}}r}
\textbf{Senior Software Engineer II}&Citrix ByteMobile\\
2010-present&Champaign, IL\\
\end{tabular*}
\begin{packed_item}
\item
Design, implement and maintain features for dynamic optimization of Internet
video.
\item
Rearchitect software modules for increased performance (30\% reduction
in latency in one case), extensibility and stability.
\item
Champion for improved software engineering practices including test-driven
development, continuous integration, static code analysis and randomized
black-box testing.
\item
Lead small development teams (2-3 developers) using Scrum methodologies.
\end{packed_item}


% MOBIUS
\begin{tabular*}{6.10in}{l@{\extracolsep{\fill}}r}
\textbf{Graduate Research Assistant}&Advisor: Prof. Robin Kravets\\
2003-2010&University of Illinois at Urbana-Champaign\\
\end{tabular*}
%\begin{inmargins}{0.25in}
%Participating in the Phoenix project to research delay tolerant communication in
%day-after networks - networks formed with limited or unavailable access to the
%core Internet infrastructure.
%\end{inmargins}
% TODO: make this project: challenges: solutions
\begin{packed_item}
\item
Research focus on creating protocols and software to enable sharing of idle
network resources by end-users in dynamic mobile-computing environments.
\item
Developed a congestion control protocol for intermittenly-connected networks
that improves message delivery by between 15\% and 73\% in extreme cases. 
Results are based on simulation and modeling.
\item
Created a location-based data overlay for vehicular networks which uses location
history to control data-packet routing decisions improving data query rate by
300\% - 600\% compared to traditional mechanisms.  Vehicular-network simulation
based on real urban topologies was used to perform protocol analysis.
\item
Designed a Wi-Fi authentication scheme for mobile clients based on end-user
social connections that improves resiliency and reduces authentication delays
by up to 50\% in residential settings.  Evaluated using network traces from a
large ISP and testbed deployment.
\item
Implemented a software system for Practical End-host Residential Multihoming
which enables access to multiple Wi-Fi access points to reduce network latency
by up to 15\% for short-duration flows and can improve download times by
28\%-62\% for longer-lived sessions.  Results from testbed based on OpenWRT
Linux router.
\end{packed_item}


% T-Labs
\begin{tabular*}{6.10in}{l@{\extracolsep{\fill}}r}
\textbf{Research Intern}&Deutsche Telekom Laboratories\\
Summer 2006&Berlin, Germany\\
\end{tabular*}
%\begin{inmargins}{0.25in}
%The Authentication on the Edge (AGE) project describes a new authentication
%architecture for global scale open Wi-Fi networks.
%A new paradigm for providing outdoor broadband access is emerging
%in which private entities (small businesses or residential users) open their
%private Wi-Fi access points to nearby users in exchange for access privileges at
%other users' locations.  In order for these open Wi-Fi networks to gain wide
%adoption strict access control and traceability are needed.  AGE localizes
%authentications to the edge of the network by requiring each user to carry all
%of his authentication credentials.  AGE supports a social network overlay as a
%fall back mechanism to relieve the bottleneck and single failure point of a
%central authentication server.
%\end{inmargins}
\begin{packed_item}
\item
Analyzed SMS traces to inform design of end-user-based authentication for
shared Wi-Fi networks which was implemented
for FreeRADIUS and wpa\_supplicant for Linux.
\end{packed_item}



% DPRG
\begin{tabular*}{6.10in}{l@{\extracolsep{\fill}}r}
\textbf{Graduate Research Assistant}&DPRG Research Group, Prof. Indranil Gupta\\
2004-2005&University of Illinois at Urbana-Champaign\\
\end{tabular*}
%\begin{inmargins}{0.25in}
%%The DPRG group studies design, implementation, and
%%evaluation issues for large-scale distributed systems such as peer-to-peer
%%systems, the Grid and sensor networks.
%Our design methodology for distributed protocols identifies multiple building
%blocks which compose many probabilistic protocols previously developed.  The
%methodology also specifies the composition rules by which the building blocks
%are combined into complete protocols.  The Proactive Protocol Composition
%Language is used to specify new protocols from existing C code based on this
%methodology.
%\end{inmargins}
\begin{packed_item}
\item
Researched methodologies for composing and designing distributed protocols using
multiple building blocks which compose many probabilistic protocols previously
developed.
\item
Developed the Proactive Protocol Composition Language (PPCL), a toolkit 
for composing distributed protocols from existing C code.
\end{packed_item}
\iffalse
\fi

\pagebreak
\section{Professional Experience (cont.)}

% NCSA
\begin{tabular*}{6.10in}{l@{\extracolsep{\fill}}r}
\textbf{Graduate Research Programmer}&
National Center for Supercomputing Applications\\
2004-2005&Urbana, IL\\
\end{tabular*}
%\dates{2004-2005}
%\employer{NCSA}
%\location{Urbana, IL}
%\title{Graduate Research Programmer}
%\begin{position}
%\begin{inmargins}{0.15in}
%The Distributed Applications Support Team at NCSA was part of the National
%Laboratory of Advanced Network Research.  The group developed tools for
%monitoring and managing high performance networks.
%\end{inmargins}
\begin{packed_item}
\item
Implemented and maintained several Java packages including network test suite
and core analysis engine
as part of advanced network monitoring and analysis tool
(NLANR Network Performance Advisor) that allows end-users to diagnose common
Internet problems based on gathered metrics.

\end{packed_item}
%\end{position}





\iffalse
% Pablo
\begin{tabular*}{6.10in}{l@{\extracolsep{\fill}}r}
\textbf{Graduate Research Assistant}&Pablo Research Group, Prof. Dan Reed\\
2003-2004&University of Illinois at Urbana-Champaign\\
\end{tabular*}
%\begin{inmargins}{0.25in}
%The Pablo research group researched areas of high performance computing
%including performance characterization, high speed I/O, and Grid monitoring.
%\end{inmargins}
\begin{packed_item}
\item
Researched temporal performance monitoring in the Grid as part of the GrADS
project.
\end{packed_item}
\fi

%\pagebreak
% Sun
\begin{tabular*}{6.10in}{l@{\extracolsep{\fill}}r}
\textbf{Member Technical Staff, Java}&Sun Microsystems\\%, JavaSoft\\
2001-2003&Santa Clara, CA\\
\end{tabular*}
%\dates{2001-2003}
%\employer{Sun Microsystems}
%\location{Santa Clara, CA}
%\title{Member Technical Staff, JavaSoft}
%\begin{position}
\begin{packed_item}
%\renewcommand{\labelitemi}{$-$}
\item
Processed customer bug reports regarding problems
in the Java Development Kit including core Java, Web Start, JSSE and the
Java compiler.
\item
Developed internal tools using
JSP, Java servlets, Apache Struts, SQL and perl.
\end{packed_item}
%\end{position}



% Bosch
\begin{tabular*}{6.10in}{l@{\extracolsep{\fill}}r}
\textbf{Undergraduate Intern}&
Robert Bosch GmbH FV/FLI (R\&D/Systems Laboratory)\\
Summer 2000&Stuttgart, Germany\\
\end{tabular*}
\begin{packed_item}
\item
Implemented prototype steer-by-wire system in C based on CAN protocol.   
%The
%steer-by-wire system attempts to obsolete the traditional steering
%column designs with a completely electronic replacement.  Challenges include
%fault tolerance and message priority.
\end{packed_item}
\iffalse
\fi


%
% publications
%
\section{Selected Publications}
\begin{packed_enum}
\item
\textbf{Nathanael Thompson}, Riccardo Crepaldi and Robin Kravets. ``Locus: A
Location-based Data Overlay For Disruption-Tolerant Networks'', 
\textit{Proceedings ACM CHANTS}, Chicago, USA, September, 2010.

\item
\textbf{Nathanael Thompson}, Samuel C. Nelson, Mehedi Bakht, Tarek Abdelzaher
and Robin Kravets.  ``Retiring Replicants: Congestion Control for
Intermittently-Connected Networks'', \textit{Proceedings IEEE
INFOCOM}, San Diego, USA, March, 2010.

\item
\textbf{Nathanael Thompson} and Robin Kravets.  ``Poster Abstract:
Understanding and Controlling Congestion in Delay Tolerant Networks'',
\textit{SIGMOBILE Mob. Comput. Commun. Rev.}, vol 13, Issue 3,
January 2010.

\item
\textbf{Nathanael Thompson}, Petros Zerfos, Robert Sombrutzki,
Jens-Peter Redlich and
Haiyun Luo.  ``100\% Organic: Design and Implementation of
Self-Sustaining Cellular Networks'', \textit{Proceedings ACM
HotMobile}, Napa Valley, USA, February, 2008.

\item
\textbf{Nathanael Thompson}, Haiyun Luo, Petros Zerfos, Jatinder Singh,
Zuoning Yin.  
``Extended Abstract: Authentication on the Edge - Distributed Authentication
for a Global Open Wi-Fi Network'', \textit{Proceedings ACM MobiCom},
Montr\'eal, Canada, September, 2007.
%Montr�al, Canada, September, 2007

\item
Ercan Ucan, \textbf{Nathanael Thompson}, Indranil Gupta.
``A PiggyBacking Approach to
Reduce Overhead in Sensor Network Gossip'', \textit{Proceedings ACM MIDSENS},
Newport Beach, USA, November, 2007.

\item
\textbf{Nathanael Thompson}, Guanghui He, Haiyun Luo.
``Flow Scheduling for End-host
Multihoming,'' \textit{Proceedings IEEE INFOCOM}, Barcelona,
Spain, April, 2006.

\iffalse
\item
Indranil Gupta, Steve Ko, \textbf{Nathanael Thompson}, Mehwish Nagda,
Christo F.  Devaraj, Rams�s Morales, Jay A. Patel. ``A Case for Methodology
Research in Self-* Distributed Systems,'' LNCS 3460, \textit{Self-Star
Properties in Complex Information Systems (eds: O. Babaoglu et al)},
pages 260-272, 2005.
\fi

\end{packed_enum}


%
% Patents
%
\section{Patents}
\begin{packed_enum}
\item
``Method and System for Distributed, Localized
Authentication in the Framework
of 802.11'', Petros Zerfos, Jatinder Singh, Marcin Solarski, Pablo Vidales,
\textbf{Nathanael Thompson}, Haiyun Luo, US Patent No. 8,307,414 B2.
\end{packed_enum}


\iffalse
%
% tech reports
%
\section{Technical Reports}
\begin{packed_enum}

\item
\textbf{Nathanael Thompson}, Haiyun Luo,
``PERM: A Collaborative System for Residential
Internet Access,'' \textit{Technical Report UIUCDCS-R-2006-2751}, University of
Illinois, July, 2006.

%\item
%Nathanael Thompson, Ercan Ucan, Indranil Gupta, ``Scheduling of Multi-Stream
%Gossip Systems,'' \textit{Technical Report UIUCDCS-R-2006-2726}, University of
%Illinois, May, 2006

\item
\textbf{Nathanael Thompson}, Indranil Gupta, Kenneth Birman.
``A Composition Methodology
for Designing Pro-active Distributed Protocols,'' \textit{Technical Report
UIUCDCS-R-2004-2490}, University of Illinois, October, 2004.

\end{packed_enum}
\fi


%
% posters
%
\iffalse
\section{Posters}
\begin{packed_enum}
\item
\textbf{Nathanael Thompson} and Robin Kravets.
``Understanding and Controlling Congestion
in Delay Tolerant Networks,'' \textit{Poster, ACM MobiCom}, San Francisco, CA,
September, 2008.

\item
Guanghui He, \textbf{Nathanael Thompson}, Haiyun Luo.
``Individual User WLAN Traffic
Analysis,'' \textit{Poster, ACM MobiHoc}, Champaign, Illinois, May, 2005.
\end{packed_enum}
\fi

\iffalse
%
% Thesis
%
\section{Undergraduate Honors Thesis}
\begin{packed_enum}
\item
\textbf{Nathanael Thompson}, Thomas W. Doeppner, Maurice Herlihy,
``Group Membership in
Ad Hoc Wireless Networks,'' \textit{Undergraduate Honors Thesis, Brown
University}, Providence, RI, May, 2001.
\end{packed_enum}
\fi



%
% Talks
%
\section{Talks}
\begin{packed_enum}
\item
``Congestion Control for Intermittently-Connected
Networks'', INFOCOM, March 2010.
\item
``Design and Implementation of
Self-Sustaining Cellular Networks'', HotMobile, Feb.\ 2008.
\item
``Flow Scheduling for End-host Multihoming'', INFOCOM, April 2006.
\end{packed_enum}


%
% Media Coverage
%
\section{Media Coverage}
\begin{packed_item}
\item
Brown, Bob.
``Residential Wi-Fi sharing made easy.'' NetworkWorld.com. 27 April 2006
\\$<$http://www.networkworld.com/news/2006/042706-sharing-wi-fi.html$>$.
\end{packed_item}

%
% awards
%
\section{Awards and Activities}
\begin{packed_item}
\item
Recipient 2008 ACM MobiCom travel grant.
\item
Recipient of 2006 Deutsche Telekom PAM (PhD - Advisor - Mentor) Fellowship.
\item
Member ACM SIGMOBILE.
\end{packed_item}

%
% teaching
%
\iffalse
\section{Teaching Experience}
\begin{tabular*}{6.10in}{l@{\extracolsep{\fill}}r}
\textbf{Graduate Teaching Assistant}&University of Illinois at Urbana-Champaign\\
2006&Urbana, IL\\
\end{tabular*}
\begin{packed_item}
\item
Assistant for large undergraduate programming course and graduate level
networking class.
\item
Led weekly discussion sections and held office hours.
\item
Designed student homeworks, programming assignments and exams.
\end{packed_item}
\fi


\iffalse
\begin{tabular*}{6.10in}{l@{\extracolsep{\fill}}r}
\textbf{Graduate Teaching Assistant}&CS241: System Programming\\
Fall 2006&University of Illinois at Urbana-Champaign\\
\end{tabular*}
%\begin{inmargins}{0.25in}
%CS241 is an undergraduate level course teaching introductory systems
%programming and Operating Systems concepts.
%\end{inmargins}
\begin{packed_item}
\item
Led two weekly discussion sections of 20 students each.
\item
Designed and graded student programming assignments.
\item
Wrote student homework and co-wrote exams.
\end{packed_item}

\begin{tabular*}{6.10in}{l@{\extracolsep{\fill}}r}
\textbf{Graduate Teaching Assistant}&CS438: Computer Networks\\
Spring 2006&University of Illinois at Urbana-Champaign\\
\end{tabular*}
%\begin{inmargins}{0.25in}
%CS438 is a mixed senior / graduate level class teaching introductory concepts in
%networking.  Students also complete advanced systems programming assignments.
%\end{inmargins}
\begin{packed_item}
\item
Developed homework and programming assignments.
\item
Delivered several lectures and held office hours.
\end{packed_item}
\fi

%
% tech skills
%
\section{Programming Languages}
%\begin{ncolumn}{2}
\begin{tabular*}{6.10in}{ll}
Primary: C++, Java, C, perl, bash&Secondary: python, JavaScript\\
%Debugging environments:&dbx, gdb\\
%Other Technologies:&subversion (CVS), LaTeX, vi\\
%Spoken Languages:& English (native), German (excellent), French (conversational), \\ & Mandarin (basic spoken)\\

\end{tabular*}
%\end{ncolumn}


%
% references
%
\iffalse
\section{References}
\begin{tabular*}{6.10in}{l@{\extracolsep{\fill}}r}
References available upon request.&\\
\end{tabular*}
\fi

\end{resume}
\end{document}
