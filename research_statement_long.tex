\documentclass[10pt]{article}

\topmargin      0.0in
\headheight     0.0in
\headsep        0.0in
\oddsidemargin  0.0in
\evensidemargin 0.0in
\textheight     9.0in
\textwidth      6.5in

\author{Nathanael Thompson}
\title{Research Statement}

\begin{document}
%\maketitle
\begin{center}
\begin{tabular}{lr}
\textbf{\Large{Research Statement}} \hfill & \hfill \textbf{\Large{Nathanael Thompson}} \\
\hline
\end{tabular}
\end{center}

I like mobile and ubiquitous computing with a focus on wireless networking.


\noindent \textbf{Vision: User Created Networks} \\
% Wireless devices are everywhere.
The development of wireless communication devices that operate in the
unlicensed spectrum has created a potential utopia for on-demand networking.
% Lots of opportunities for dynamic networks (unlicnesed and spontaneous)
With millions of Wi-Fi enabled devices in the hands of end-users, the
potential networking opportunities are abundant.
Any person, anywhere can create a data network and communicate with other people
nearby.  
% Benefits.
Because these user-created networks (UNs) are deployed by end-users themselves,
the monetary cost to
expand network coverage is close to zero.  User-created networks are an
efficient method to increase the network bandwidth and availability
for all.

% Challenges in these networks coming from dynamism:
Networking in user-created scenarios is challenged by the dynamics and
heterogeneity of the network.
% no pre-configuration (resource detection and utilitization)
Connectivity is limited due to mobility, fluctuations in the
wireless channel, unpredictable
node availability and unplanned node deployment.
The result is that devices may be subjected to frequent periods of
disconnectivity which means resource availability is very dynamic.
% limited resources (demand participation of devices themselves)
To improve the throughput and reliability of communication in user-created
networks, multiple end-users can
collaboratively pool their resources and carry each other's
traffic.



% Existing solutions won't work because of lack of reliable access to existing
% infrastructure and performance considerations.
Any solutions to enable cooperative networking for user-created networks
cannot rely on supporting infrastructure to achieve their goal.
The user-created network is deployed by end-users who only manage
their own set of devices at the edge of the network and typically are
not able to deploy services in the Internet.  In addition, users are likely
unwilling to pay for user-created networking, meaning the incentive to
third-parties is very low.

% My research realizes we need local solutions to the above problems.
%While some of these requirements have been studied before, existing research has
%focused on targeted domains and is largely inapplicable to the user-created
%network environment.
In my research I design protocols and algorithms for user-created networking
that only use local
information to dynamically adapt to the changing resource and communication
constraints of the network.  These solutions operate without the need of
external infrastructure, only using it to enhance performance if available.
I use real-world implementations and simulation results to demonstrate the
effectiveness of these approaches. \\



\noindent \textbf{Congestion Control for Intermittently-Connected Networks}
%In some cases user-created networks are delay tolerant.
In many cases user-created networks will be only intermittently-connected due to
human mobility and potentially sparse deployment.
These intermittently-connected networks (ICNs)
typically lack stable end-to-end paths.  
In order to improve the
delivery rates of the networks, new store-carry-and-forward protocols have been
proposed which often use message replication as a forwarding mechanism.  
% Store-carry-and-forward networks have three constraints.  Storage, bandwidth
% and energy.
Message
replication is effective at improving delivery, but given the limited
resources of ICN nodes, such as buffer space, bandwidth and energy, as well as
the highly dynamic nature of these networks, replication can easily overwhelm
node resources.  Over-replication leads to network congestion and reduced
delivery.  

Given a number of message priortization schemes which effectively address the
bandwidth constraint, my research focuses on the buffer constraint.  By managing
the replication rate, congestion can be avoided and message delivery maximized.
Traditional sender-based congestion control does not succeed in this
environment.  Intermittent
connectivity, path diversity, high loss rates and long response times
make it difficult for the sender to react to congestion in a timely
and effective manner.  Also, replication is performed at every node
that has a replica of the message.  Therefore, a sender cannot know
how every node that will receive a replica of its messages should
react.  
Instead of sender-based congestion control, congestion control in ICNs
becomes a problem of managing replication at every node in the
network.

% CC provides node-based, localized congestion control for ICNs.
In response to this challenge, I have developed a novel
node-based replication management algorithm which addresses buffer congestion by
dynamically limiting the
replication a node performs during each encounter~\cite{thompson:cc:infocom10}. 
Individual nodes estimate
current congestion levels by locally sampling network replications and drops.
The insight for the
algorithm comes from a stochastic model of message delivery in ICNs with
constrained buffer space.  
Simulation results show that our algorithm is
effective, nearly tripling delivery rates in some scenarios, and
imposes no or little overhead. \\

% Extended with energy.
%On-going work extends replication limiting to include energy management into the
%per-encouter limit.


\noindent \textbf{Flow Scheduling to Aggregate Available Bandwidth}
% Performance of user-created networks depends on smart usage of dynamically
% changing resources.
High performance communication in user-created networks is only achievable if
the dynamically available resources are effectively utilized.  An important
resource is the available bandwidth to the Internet.  Publically-accessible
802.11 basestations, installed in homes or businesses, are widely available in
most areas.  In addition, wide-area enabled devices with 3G or 4G connections
can also become Internet gateways to service nearby clients.
By connecting simultaneously to more than one gateway either using software or
hardware techniques, an end-user can achieve higher bandwidth and higher
connection resiliency.

% aggregating in UN is different than before (related work)
Previous research to utilize multiple Internet connections cannot be deployed in
the user-created network because of the lack of reliable infrastructure support.
Exist solutions require either protocol support from the end-hosts in the
Internet or on the LAN or interact with proxy servers accessible from any
edge network.  Because the user-created network cannot rely on infrastructure
support, a self-contained solution is necessary.


%PERM monitors available wireless basestations and schedules traffic based on
%learned user behavior.
To operate without external support, an end-host must perform flow scheduling
across the multiple available Internet connections.  I have designed and
implemented the PERM system for performing end-host flow
scheduling~\cite{thompson:perm:infocom06}.
Based on our study of individual end-user networking
behaviors~\cite{he:traffic:mobihoc05}, PERM 
exploits its proximity to the end-user to improve scheduling effectiveness
by adopting automated on-line learning of the
end-user's traffic patterns to make the best match between flows and available
Internet connections.  
By monitoring the available bandwidth on each connection, PERM can intelligently
schedule flows to reduce latency and increase throughput.
Evaluation in a residential deployment with an open source wireless
router and Linux and Windows clients shows 
significantly improved reliability, throughput and response times.  \\


\noindent \textbf{AGE}
% lack of trust and privacy.
Once end-users begin carrying traffic for each other in a user-created network,
trust becomes an issue.  
% Authentication is needed to create trust.
To ensure accountability and authorization, mutual authentication between users
is needed.
%Centralized trust to avoid bootstrap and agreement.
Because of the diverse set of users in the end-user network, including end-users
from different geographic areas, businesses and ISPs, a centralized trust
management is needed.
% Online authentication takes too long.
Despite the benefit of a central trust source, online centralized authentication
is not practical in user-created networks.  Existing authentication schemes are
either overwhelmed by the large size of the
user created network in terms of number of clients as well as physical
deployment or degrade in performance too much from the network delay and
possible loss of connection when
communicating with Internet-based servers from the user-created network.

% AGE is offline authentication for user-created networks.
In collaboration with researchers at Deutsche Telekom T-Labs, I have developed
a localized and distributed
authentication method for open Wi-Fi networks that can be used between any
devices in the user-created
network~\cite{thompson:age:mobicom07, thompson:organic:hotmobile08}.
The protocol
adapts to the variability and unreliability of these networks by using 
certificate-based authentication, the distribution of certificate revocation
list segments to all entities, and the self organization of access points into a
social look-up network.  These techniques allow for offline authentication of
centrally administered credentials.  Devices only need to sync with the central
trust store infrequently.
The protocol has been implemented as an extension to the FreeRADIUS server and
shown to reduce overall
authentication delays are by as much as 71\% compared to existing
authentication schemes applied in the same scenario.   \\


\noindent \textbf{Ongoing/Future Work}
Many services for user-created networks are location-based and therefore would
benefit from location-aware routing.
Disconnected.  Many services benefit from geographic routing.  Existing
geographic routing does not do multicopy and often assumes a lot of information
from network.

Building from previous work, we are designing a node-based multi-copy geographic
forwarding protocol for VANETs.  The protocol decides which messages to copy
during an ecounter based on the geographic information available at each node.
Geographic information speeds spread through the network towards the
destination.  Multicopy allows for multiple path exploration and increases the
chances that a mobile node will be found. \\


\noindent \textbf{Other research}
I have also done research on scheduling gossip streams to reduce
overhead in emerging multiple-source, multiple-recipient message distribution
systems~\cite{ucan:piggyback:midsens07} as well as on self-* distributed system
design methodologies~\cite{gupta:methodology:lncs05}.

\footnotesize
\bibliographystyle{plain}
\bibliography{nat-pubs}
\end{document}
