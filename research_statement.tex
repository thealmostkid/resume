\documentclass[10pt]{article}

\topmargin      0.0in
\headheight     0.0in
\headsep        0.0in
\oddsidemargin  0.0in
\evensidemargin 0.0in
\textheight     9.0in
\textwidth      6.5in

\author{Nathanael Thompson}
\title{Research Statement}

\begin{document}
%\maketitle
\begin{center}
\begin{tabular}{lr}
\textbf{\Large{Research Statement}} \hfill & \hfill \textbf{\Large{Nathanael Thompson}} \\
\hline
\end{tabular}
\end{center}

What is future of mobile computing?

One, fat pipes to the cloud.  How to access cloud services from mobile device.

Second, a less explored future is direct device to device communication.  It
complements the first development by expanding coverage, potentially reducing
network load on backhaul links and facilitating the deployment of new services
and applications for end-users.

Because of the proliferation of devices in unlicensed spectrum and decreasing
cost of computing, there are tons of devices out there including phones, cars,
sensors, laptops, private and public wi-fi basestations.

The resources of these devices can be pooled to increase the available
communication, but also the environmental awareness of end-users.

The challenges of harnessing this unused resource come from the the fact that
the devices are all end-user devices.  Compared to Internet-based servers and
routers, the devices are very limited in resources.  In addition, each device
or small set of devices is managed by a different user making configuration and
management difficult.

From a networking perspective, these challenges impact the performance of the
network in terms of throughput/delivery rate and delay.

Existing networking solutions to improve network performance are not applicable
in the user-created network scenario.  Because the networks are deployed by
end-users themselves, no end-to-end solutions can be used.  It is also unlikely
that end-users can deploy services in the Internet to meet their needs, and
because the network is user-created there is no financial incentive for third
parties to support the network.

Therefore, new solutions are needed that do not rely on any infrastructure-based
support.  In my research I have developed protocols and algorithms that rely
only on local information to dynamically adapt to the changing resources of UN.


Congestion control for DTNs:
In many cases the UN will be only intermittently connected.
Store-carry-forward.  Uses replication.  Overwhelms or underwhelms.

Replication management is needed to find correct balance.  Our replication
management is node-based, etc.


Integrating private wi-fi access points:
One advantage of the un is that it can offload traffic from the WWAN backhaul
links.  In order to do so available bandwidth from wifi needs to be
incorporated.  
The available throughput is limited and delay might be bad depending on access
point.
However, users are often presented with multiple access points in range.
Need to aggregate them.  PERM is a completely-local user aggregation framework.
It connects to multiple access points and schedules traffic based on user
behavior by doing flow-scheduling.

Due to a number of high profile security violations and a better educated
populace, users are becoming less and less willing to leave their access points
open for public access.  However, as several projects have demonstrated, users
are willing to share their bandwidth with other people.  A global authentication
mechanism is needed to authenticate devices in the UN.  Must be central because
of the huge scope.  However, online centralized authentication can incur very
high delay (bad when connections are short), might experience outages and
requires large resources.  To address these challenges I have developed
authentication on the edge in collaboration with researchers at Deutsche-Telekom
Laboratories.
It's offline and cool.


Ongoing Work
One promising application for UN is location-based services.  To best support
location-based services, location-based routing is needed.  It scales and avoids
expensive translations.  Solutions exist for ad hoc networks.  However,
intermittent connectivity in UN breaks existing solutions.  Simply adding
store-carry-and-forward is insufficient.  Examining new routing techniques.





Two directions for mobile computing.  First is fast wireless connections to the
cloud through 4G, LTE, Wi-Max providing access to Internet services and data.
The evolution of mobile broadband means better services to the end-user.  Yet,
there remains a second direction for mobile computing which has yet to be fully
realized.  
Wireless networking and decrease in cost of computing has resulted in an
explosion of mobile computers such as handsets/smart phones, vehicles, embedded
sensors, mesh networks, private wi-fi access points and laptop computers,
leading to an increasing abundance of direct
device-to-device communication opportunities.

While the device-to-device communication can extend network coverage
and allow new services and applications to be delivered to end-users, such
user-created networks (UNs) are deployed on end-user devices by end-users
themselves which creates challenges due to the limited resources and management
available.

I like mobile and ubiquitous computing with a focus on wireless networking.


\noindent \textbf{Vision: User Created Networks} \\
% Wireless devices are everywhere.
The explosion of wireless communication devices that operate in the
unlicensed spectrum has created a potential utopia for on-demand networking.
% Lots of opportunities for dynamic networks (unlicnesed and spontaneous)
With millions of Wi-Fi enabled devices in the hands of end-users, the
networking opportunities are abundant.
Any person, anywhere can create a data network and communicate with other people
nearby.  
% Benefits.
Because these user-created networks (UNs) are deployed by end-users themselves,
the monetary cost to
expand network coverage is close to zero, making user-created networks an
efficient method to increase the network bandwidth and availability
for everyone.

User-created networks will only truly be useful if multiple end-users 
collaboratively pool their resources and carry each other's
traffic.  While the co-operative approach expands the available resources for
each user, it also places a burden on the limited resources of the end-user
devices.
% Challenges in these networks coming from dynamism:
In addition, communication in user-created scenarios is challenged by the
dynamics and heterogeneity of the network arising
from node mobility, fluctuations in the
wireless channel, unpredictable
node availability and unplanned node deployment.
The result is that devices may be subjected to frequent periods of
disconnectivity which further increases the resource demand.
% limited resources (demand participation of devices themselves)
% no pre-configuration (resource detection and utilitization)


% Existing solutions won't work because of lack of reliable access to existing
% infrastructure and performance considerations.
Any solutions to alleviate the burden of
co-operative networking for user-created networks
cannot rely on supporting infrastructure to achieve their goal.
The user-created network is deployed by end-users who only manage
their own set of devices at the edge of the network and typically are
not able to deploy services in the Internet.  In addition, users are likely
unwilling to pay for user-created networking, meaning the incentive for
third-parties to support user-created networks is low.

% My research realizes we need local solutions to the above problems.
%While some of these requirements have been studied before, existing research has
%focused on targeted domains and is largely inapplicable to the user-created
%network environment.
In my research I have designed protocols and algorithms for user-created
networks that only use local
information to dynamically adapt to the changing resource and communication
constraints of the network.  These solutions operate without the need of
external infrastructure, only using it to enhance performance if available.
I use real-world implementations and simulation results to demonstrate the
effectiveness of these approaches. \\



\noindent \textbf{Congestion Control for Intermittently-Connected Networks}
%In some cases user-created networks are delay tolerant.
In many cases user-created networks will be only intermittently-connected due to
human mobility and areas of sparse deployment.
Such intermittently-connected networks (ICNs)
lack stable end-to-end paths rendering existing routing techniques ineffective.
In order to improve the
delivery rates of the networks, new store-carry-and-forward protocols have been
proposed which often use message replication as a forwarding mechanism.  
% Store-carry-and-forward networks have three constraints.  Storage, bandwidth
% and energy.
Message
replication is effective at improving delivery, but given the limited
resources of ICN nodes, such as buffer space, bandwidth and energy, 
%as well as the highly dynamic nature of these networks, 
replication can easily overwhelm
node resources.  Over-replication leads to network congestion and reduced
delivery while under-replication can lead to missed delivery opportunities.  

By managing the replication rate in the network, an effective balance can be
found that minimizes congestion of resources and improves delivery rates.
Traditional sender-based congestion control does not succeed in this
environment.  Intermittent
connectivity, path diversity, high loss rates and long response times
make it difficult for the sender to react to congestion in a timely
and effective manner.  Also, a sender cannot know
how every node that will receive a replica of its messages should
react.  
Instead of sender-based congestion control, congestion control in ICNs
becomes a problem of managing replication at every individual node in the
network.

% CC provides node-based, localized congestion control for ICNs.
In response to this challenge, I have developed a novel
node-based replication management algorithm which prevents congestion by
dynamically limiting the
replication a node performs during each encounter~\cite{thompson:cc:infocom10}. 
The algorithm operates locally at individual nodes without requiring global
knowledge.  Because each node runs independently, the algorithm can easily
respond to different environments in the network where the congestion might be
higher or lower.
To do this, individual nodes estimate
current congestion levels by locally sampling network replications and drops.
Using the sample, each node sets a limit on the number of messages it will
transfer during the next encounter.
Simulation results show that our algorithm is
effective, nearly tripling delivery rates in some scenarios, and
imposes no or little overhead. \\

% Extended with energy.
%On-going work extends replication limiting to include energy management into the
%per-encouter limit.


\noindent \textbf{Flow Scheduling to Aggregate Available Bandwidth}
% Performance of user-created networks depends on smart usage of dynamically
% changing resources.
High performance communication in user-created networks is only achievable if
the dynamically available resources are effectively utilized.  An important
resource in user-created networks
is the available outgoing bandwidth to the Internet.  Publically-accessible
802.11 basestations, installed in homes or businesses, are widely available in
most areas.  In addition, WWAN enabled devices with 3G or 4G connections
can also become Internet gateways to service nearby clients.
However, the available bandwidth through each gateway is limited due to
contention from other users, the sharing policies of the owner or the backhaul
link itself.

By connecting simultaneously to more than one gateway either using software or
hardware techniques, an end-user can achieve higher bandwidth and higher
connection resiliency.
% aggregating in UN is different than before (related work)
Previous research to utilize multiple Internet connections cannot be deployed in
the user-created network because of the lack of reliable infrastructure support.
Existing solutions require either protocol support from the end-hosts in the
Internet or LAN or interact with proxy servers.
A dynamically deployed network created by end-users cannot expect such
infrastructure support to be available.


%PERM monitors available wireless basestations and schedules traffic based on
%learned user behavior.
To operate without external support, an end-host must perform flow scheduling
across the multiple available Internet connections.  I have designed and
implemented the PERM system for performing end-host flow
scheduling~\cite{thompson:perm:infocom06}.
Based on our study of individual end-user networking
behaviors~\cite{he:traffic:mobihoc05}, PERM 
exploits its proximity to the end-user to improve scheduling effectiveness
by adopting automated on-line learning of the
end-user's traffic patterns to make the best match between flows and available
Internet connections.  
By monitoring the available bandwidth on each connection, PERM can intelligently
schedule flows to reduce latency and increase throughput.
Evaluation in a residential deployment with the open source OpenWRT wireless
router and Linux and Windows clients shows 
significantly improved reliability, throughput and response times for typical
end-user traffic.  \\


\noindent \textbf{Authentication on the Edge}
% lack of trust and privacy.
Once end-users begin carrying traffic for each other in a user-created network,
trust becomes an issue.  
% Authentication is needed to create trust.
To ensure accountability and authorization, mutual authentication between users
is needed.
%Centralized trust to avoid bootstrap and agreement.
Because the user-created network can include users of different organizations,
private APs and ISP networks,
centralized trust management is needed.
% Online authentication takes too long.
However, online centralized authentication
is not practical in user-created networks.  Existing authentication schemes are
either overwhelmed by the large size of the
user created network in terms of number of clients as well as physical
deployment or degrade in performance too much from the network delay and
possible loss of connection when
communicating with Internet-based servers from the user-created network.

% AGE is offline authentication for user-created networks.
In collaboration with researchers at Deutsche Telekom T-Labs, I have developed
a localized and distributed
authentication method for open Wi-Fi networks that can be used between any
devices in the user-created
network~\cite{thompson:age:mobicom07, thompson:organic:hotmobile08}.
The protocol
adapts to the variability and unreliability of these networks by using 
certificate-based authentication, the distribution of certificate revocation
list segments to all entities, and the self organization of access points into a
social look-up network.  These techniques allow for offline authentication of
centrally administered credentials between devices.  Devices only need to sync
with the central
trust store infrequently based on the user-configured freshness threshold.
The protocol has been implemented as an extension to the FreeRADIUS server and
shown to reduce overall
authentication delays are by as much as 71\% compared to existing
authentication schemes applied in the same scenario.   \\


\noindent \textbf{Ongoing/Future Work}
Many services for user-created networks are location-based and therefore would
benefit from location-aware routing.
Disconnected.  Many services benefit from geographic routing.  Existing
geographic routing does not do multicopy and often assumes a lot of information
from network.

Building from previous work, we are designing a node-based multi-copy geographic
forwarding protocol for VANETs.  The protocol decides which messages to copy
during an ecounter based on the geographic information available at each node.
Geographic information speeds spread through the network towards the
destination.  Multicopy allows for multiple path exploration and increases the
chances that a mobile node will be found. \\


\noindent \textbf{Other research}
I have also done research on scheduling gossip streams to reduce
overhead in emerging multiple-source, multiple-recipient message distribution
systems~\cite{ucan:piggyback:midsens07} as well as researching design
methodologies for self-* distributed systems~\cite{gupta:methodology:lncs05}.

\footnotesize
\bibliographystyle{plain}
\bibliography{nat-pubs}
\end{document}
